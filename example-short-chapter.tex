%%%% Time-stamp: <2012-08-20 17:41:39 vk>

%% example text content
%% scrartcl and scrreprt starts with section, subsection, subsubsection, ...
%% scrbook starts with part (optional), chapter, section, ...
\chapter{Example Chapter}

This is my text with an example Figure~\ref{fig:example} and example
citation~\cite{StrunkWhite} or \textcite{Bringhurst1993}. And there is another
\enquote{citation} which is located at the bottom\footcite{tagstore}. Here we have an acronym  \newacronym{acr:example}{ECB}{Dance the ECB} \gls{acr:example}. And then there is a vector graphics done with \href{http://tug.org/PSTricks/}{\texttt{PSTricks}}, Figure~\ref{fig:pstricks}.

\myfig{TU_Graz_Logo}%% filename in figures folder
  {width=0.1\textwidth,height=0.1\textheight}%% maximum width/height, aspect ratio will be kept
  {Example figure.}%% caption
  {}%% optional (short) caption for table of figures
  {fig:example}%% label
  
\begin{figure}[h]
	\begin{pdfpic}
		\begin{pspicture}(-5.25,-5.25)(5.25,5.25)%
			\pscircle*[linecolor=cyan]{5}
			\psgrid[subgriddiv=0,gridcolor=lightgray,gridlabels=0pt]
			\Huge\sffamily\bfseries
			\rput(-4.5,4.5){A} \rput(4.5,4.5){B}
			\rput(-4.5,-4.5){C}\rput(4.5,-4.5){D}
			\rput(0,0){pdftricks}
			\rmfamily
			\rput(0,-3.8){PSTricks}
			\rput(0,3.8){\LaTeX}
		\end{pspicture}
	\end{pdfpic}
\caption{PSTricks Inside.}
\label{fig:pstricks}
\end{figure}

Now you are able to write your own document. Always keep in mind: it's
the \emph{content} that matters, not the form. But good typography is
able to deliver the content much better than information set with bad
typography. This template allows you to focus on writing good content
while the form is done by the template definitions.


%% vim:foldmethod=expr
%% vim:fde=getline(v\:lnum)=~'^%%%%\ .\\+'?'>1'\:'='
%%% Local Variables: 
%%% mode: latex
%%% mode: auto-fill
%%% mode: flyspell
%%% eval: (ispell-change-dictionary "en_US")
%%% TeX-master: "main"
%%% End: 
